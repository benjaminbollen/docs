\documentclass[12pt,a4paper, twocolumn]{article}

\usepackage[english]{babel}
\usepackage{amsmath}
\usepackage{tikz}
\usetikzlibrary{calc,positioning,shadows.blur,decorations.pathreplacing}
\usepackage{etoolbox}


\title{{\sc OST: Heterogeneous Composition of Blockchains for Multiprocess Decentralised Computation}}
\author{OpenST Foundation Ltd.}
\date{December 29, 2017, v0.9}

\begin{document}

\maketitle

\section{Introduction}

The advent of open, decentralised blockchain systems has demonstrated that consensus engines driven by market forces can support an ongoing, crypto-economically secured transaction ledger.  Most notably Bitcoin and Ethereum have successfully implemented a respective stateless asset and stateful account model on top of Proof-of-Work consensus engines.  Both systems charge a variable transaction fee to include and process transactions into the ledger and to dynamically modulate network demand given  constraint transaction throughput capacity.  More strongly so, it has been shown [VerifiersDilemma ref] that for Proof-of-Work the computational \emph{efficiency} must be low, i.e. the amount of useful computation done in a set of transactions must be small compared to the amount of hashing power expended to secure that set into a block.
Both Bitcoin and Ethereum implement a limit for the useful computational expenditure a valid block can contain measured in bytes or gas respectively.  As Proof-of-Work maximises its security by maximising the amount of computational effort spent, both systems increase the transaction throughput capacity over time.  The cost for the same 

[TBC]

\section{Heterogeneous composition}

Consider a Byzantine fault tolerant (BFT) blockchain $E$. Let $\Sigma$ be an open group of sealers defined on the BFT blockchain $E$.  Each sealer $\sigma$ has put forward a stake on the BFT blockchain $V$. Consider votes on a couple $(t_i, e_j)$ where $t_i$ is a transaction that consumes the preceding transaction $t_{i-1}$ and $e_j$ is the blockhash of $V$ at height $j$.

\subsection{ostBLOCKS}

carrying forward gas model;
gaslimit paid upfront; 

\section{Cores}

Let a \emph{core} $d$ be a deterministic, finite state machine with a merklelized state root after each state transition upon which one can implement the OST protocol presented. For efficiency reasons we assume that the core groups state transitions into blocks and we will consider in this paper all cores to be a blockchain network.  A blockchain network additionally has a consensus engine with rules $r_d$ to reach consensus on a linear sequence of blocks.

Consider two cores $d_1$ and $d_2$ that each produce blocks according to their consensus rules.  The cores can produce blocks independently. Assume a node that fully verifies both cores.  The node can report the block header of $d_1$ to $d_2$ and reversely $d_2$ to $d_1$.

------
Let $v$ be a Byzantine Fault Tolerant blockchain. We denote the state space of $v$ as the direct sum of state spaces $c_0$ and $c_i$ with $i \in {1, N}$, where we consider $c_0$ as the canonical state space per construction. 
\begin{align}
	v \doteq \oplus \ c_i	
\end{align}


% tokens invariant per chain;
% OST does not transfer messages; OST is slow and spans across blocks on both chains
\section{Transposing Tokens}

%two-phased commit directional $c$, and $\bar{c}$;


% Hashlock

\subsection{Hashlock}

\section{Account Sessions}

An account allows a user to manage her devices. Devices in turn
can be used to create and revoke sessions, and redeem branded
tokens.  A session combines a session key and a set of single-use,
cryptographically linked secrets which the session key can include
to authorize new transactions on behalf of the account with.

A session has spending limits, and a single secret has smaller
spending limits. Spending limits can be a combination of amount
restrictions, expiration or frequency constraints and are enforced
by the account smart contract. The session spending limits can be
smaller than the product of the number of secrets and the secret
spending limits.

To create a session an application client privately generates
a random 256-bit seed and produces $n$ iteratively keccak256 hash
images of this seed.  The last generated image we label $s_0$,
the pre-image to $s_0$, $s_1$ and so on:
\begin{align}
	s_i = \text{keccak256}(s_{i+1}) \\
	s_n \text{ is the initial random seed} \nonumber
\end{align}
where $i \in [0, n-1]$.
The application server generates a new session key and the server
is responsible for funding the session key with Simple Token Prime
as the base token to pay for gas costs.  The application server and
the application client exchange the address for the session key and
$s_0$.  The application client generates a crypto-image with the address
and the initialisation secret $s_0$ and presents it to the user.
The application server pushes the session address and the initialisation
secret to the device for proposal.
Within Simple Token Wallet (or other wallet integrations) that
manages the device private key the user can visually verify the same
crypto-image, set spending limits and approve the session by signing
with the device key held in the wallet activating the session.

Within an active session the session key can sign transactions to spend
from the account if it presents the next pre-image secret for the session
and within the spending limits of the secret and the session.

\onecolumn
\section{Diagrams}


\tikzset{%
        brace/.style = { decorate, decoration={brace, amplitude=5pt} },
       mbrace/.style = { decorate, decoration={brace, amplitude=5pt, mirror} },
        label/.style = { black, midway, scale=0.5, align=center },
     toplabel/.style = { label, above=.5em, anchor=south },
    leftlabel/.style = { label,rotate=-90,left=.5em,anchor=north },   
  bottomlabel/.style = { label, below=.5em, anchor=north },
        force/.style = { rotate=-90,scale=0.4 },
        round/.style = { rounded corners=2mm },
       legend/.style = { right,scale=0.4 },
        nosep/.style = { inner sep=0pt },
   generation/.style = { anchor=base }
}

\newcommand\particle[7][white]{%
  \begin{tikzpicture}[x=1cm, y=1cm]
    \path[fill=#1,blur shadow={shadow blur steps=5}] (0.1,0) -- (0.9,0)
        arc (90:0:1mm) -- (1.0,-0.9) arc (0:-90:1mm) -- (0.1,-1.0)
        arc (-90:-180:1mm) -- (0,-0.1) arc(180:90:1mm) -- cycle;
    \ifstrempty{#7}{}{\path[fill=purple!50!white]
        (0.6,0) --(0.7,0) -- (1.0,-0.3) -- (1.0,-0.4);}
    \ifstrempty{#6}{}{\path[fill=green!50!black!50] (0.7,0) -- (0.9,0)
        arc (90:0:1mm) -- (1.0,-0.3);}
    \ifstrempty{#5}{}{\path[fill=orange!50!white] (1.0,-0.7) -- (1.0,-0.9)
        arc (0:-90:1mm) -- (0.7,-1.0);}
    \draw[\ifstrempty{#2}{dashed}{black}] (0.1,0) -- (0.9,0)
        arc (90:0:1mm) -- (1.0,-0.9) arc (0:-90:1mm) -- (0.1,-1.0)
        arc (-90:-180:1mm) -- (0,-0.1) arc(180:90:1mm) -- cycle;
    \ifstrempty{#7}{}{\node at(0.825,-0.175) [rotate=-45,scale=0.2] {#7};}
    \ifstrempty{#6}{}{\node at(0.9,-0.1)  [nosep,scale=0.17] {#6};}
    \ifstrempty{#5}{}{\node at(0.9,-0.9)  [nosep,scale=0.2] {#5};}
    \ifstrempty{#4}{}{\node at(0.1,-0.1)  [nosep,anchor=west,scale=0.25]{#4};}
    \ifstrempty{#3}{}{\node at(0.1,-0.85) [nosep,anchor=west,scale=0.3] {#3};}
    \ifstrempty{#2}{}{\node at(0.1,-0.5)  [nosep,anchor=west,scale=1.5] {#2};}
  \end{tikzpicture}
}


Sed ut perspiciatis, unde omnis iste natus error sit voluptatem accusantium doloremque laudantium, totam rem aperiam eaque ipsa, quae ab illo inventore veritatis et quasi architecto beatae vitae dicta sunt, explicabo. Nemo enim ipsam voluptatem, quia voluptas sit, aspernatur aut odit aut fugit, sed quia consequuntur magni dolores eos, qui ratione voluptatem sequi nesciunt, neque porro quisquam est, qui dolorem ipsum, quia dolor sit amet consectetur adipisci[ng] velit, sed quia non-numquam [do] eius modi tempora inci[di]dunt, ut labore et dolore magnam aliquam quaerat voluptatem. Ut enim ad minima veniam, quis nostrum exercitationem ullam corporis suscipit laboriosam, nisi ut aliquid ex ea commodi consequatur? Quis autem vel eum iure reprehenderit, qui in ea voluptate velit esse, quam nihil molestiae consequatur, vel illum, qui dolorem eum fugiat, quo voluptas nulla pariatur?

\begin{tikzpicture}[x=1.2cm, y=1.2cm]
  \draw[round] (-0.5,0.5) rectangle (4.4,-1.5);
  \draw[round] (-0.6,0.6) rectangle (5.0,-2.5);
  \draw[round] (-0.7,0.7) rectangle (5.6,-3.5);
\end{tikzpicture}

\begin{tikzpicture}[x=1.2cm, y=1.2cm]
  \draw[round] (-0.5,0.5) rectangle (4.4,-1.5);
  \draw[round] (-0.6,0.6) rectangle (5.0,-2.5);
  \draw[round] (-0.7,0.7) rectangle (5.6,-3.5);

  \node at(0, 0)   {\particle[gray!20!white]
                   {$u$}        {up}       {$2.3$ MeV}{1/2}{$2/3$}{R/G/B}};
  \node at(0,-1)   {\particle[gray!20!white]
                   {$d$}        {down}    {$4.8$ MeV}{1/2}{$-1/3$}{R/G/B}};
  \node at(0,-2)   {\particle[gray!20!white]
                   {$e$}        {electron}       {$511$ keV}{1/2}{$-1$}{}};
  \node at(0,-3)   {\particle[gray!20!white]
                   {$\nu_e$}    {$e$ neutrino}         {$<2$ eV}{1/2}{}{}};
  \node at(1, 0)   {\particle
                   {$c$}        {charm}   {$1.28$ GeV}{1/2}{$2/3$}{R/G/B}};
  \node at(1,-1)   {\particle 
                   {$s$}        {strange}  {$95$ MeV}{1/2}{$-1/3$}{R/G/B}};
  \node at(1,-2)   {\particle
                   {$\mu$}      {muon}         {$105.7$ MeV}{1/2}{$-1$}{}};
  \node at(1,-3)   {\particle
                   {$\nu_\mu$}  {$\mu$ neutrino}    {$<190$ keV}{1/2}{}{}};
  \node at(2, 0)   {\particle
                   {$t$}        {top}    {$173.2$ GeV}{1/2}{$2/3$}{R/G/B}};
  \node at(2,-1)   {\particle
                   {$b$}        {bottom}  {$4.7$ GeV}{1/2}{$-1/3$}{R/G/B}};
  \node at(2,-2)   {\particle
                   {$\tau$}     {tau}          {$1.777$ GeV}{1/2}{$-1$}{}};
  \node at(2,-3)   {\particle
                   {$\nu_\tau$} {$\tau$ neutrino}  {$<18.2$ MeV}{1/2}{}{}};
  \node at(3,-3)   {\particle[orange!20!white]
                   {$W^{\hspace{-.3ex}\scalebox{.5}{$\pm$}}$}
                                {}              {$80.4$ GeV}{1}{$\pm1$}{}};
  \node at(4,-3)   {\particle[orange!20!white]
                   {$Z$}        {}                    {$91.2$ GeV}{1}{}{}};
  \node at(3.5,-2) {\particle[green!50!black!20]
                   {$\gamma$}   {photon}                        {}{1}{}{}};
  \node at(3.5,-1) {\particle[purple!20!white]
                   {$g$}        {gluon}                    {}{1}{}{color}};
  \node at(5,0)    {\particle[gray!50!white]
                   {$H$}        {Higgs}              {$125.1$ GeV}{0}{}{}};
  \node at(6.1,-3) {\particle
                   {}           {graviton}                       {}{}{}{}};

  \node at(4.25,-0.5) [force]      {strong nuclear force (color)};
  \node at(4.85,-1.5) [force]    {electromagnetic force (charge)};
  \node at(5.45,-2.4) [force] {weak nuclear force (weak isospin)};
  \node at(6.75,-2.5) [force]        {gravitational force (mass)};

  \draw [<-] (2.5,0.3)   -- (2.7,0.3)          node [legend] {charge};
  \draw [<-] (2.5,0.15)  -- (2.7,0.15)         node [legend] {colors};
  \draw [<-] (2.05,0.25) -- (2.3,0) -- (2.7,0) node [legend]   {mass};
  \draw [<-] (2.5,-0.3)  -- (2.7,-0.3)         node [legend]   {spin};

  \draw [mbrace] (-0.8,0.5)  -- (-0.8,-1.5)
                 node[leftlabel] {6 quarks\\(+6 anti-quarks)};
  \draw [mbrace] (-0.8,-1.5) -- (-0.8,-3.5)
                 node[leftlabel] {6 leptons\\(+6 anti-leptons)};
  \draw [mbrace] (-0.5,-3.6) -- (2.5,-3.6)
                 node[bottomlabel]
                 {12 fermions\\(+12 anti-fermions)\\increasing mass $\to$};
  \draw [mbrace] (2.5,-3.6) -- (5.5,-3.6)
                 node[bottomlabel] {5 bosons\\(+1 opposite charge $W$)};

  \draw [brace] (-0.5,.8) -- (0.5,.8) node[toplabel]         {standard matter};
  \draw [brace] (0.5,.8)  -- (2.5,.8) node[toplabel]         {unstable matter};
  \draw [brace] (2.5,.8)  -- (4.5,.8) node[toplabel]          {force carriers};
  \draw [brace] (4.5,.8)  -- (5.5,.8) node[toplabel]       {Goldstone\\bosons};
  \draw [brace] (5.5,.8)  -- (7,.8)   node[toplabel] {outside\\standard model};

  \node at (0,1.2)   [generation] {1\tiny st};
  \node at (1,1.2)   [generation] {2\tiny nd};
  \node at (2,1.2)   [generation] {3\tiny rd};
  \node at (2.8,1.2) [generation] {\tiny generation};
\end{tikzpicture}

\end{document}