\documentclass[landscape, 11pt, svgnames]{article}

\usepackage[showframe]{geometry}
\usepackage[T1]{fontenc}
\usepackage[utf8]{inputenc}
\usepackage[english]{babel}
\usepackage{listings}
\usepackage{../../../tikz-uml}
\usepackage{amsfonts, amsmath, amsthm, amssymb}

\geometry{
  paperwidth=72cm,
  paperheight=120cm,
  margin=1cm
}

\date{\today}
\title{OpenST Protocol sequence diagrams v0.9.3}
\author{Benjamin Bollen}

\lstdefinelanguage{tikzuml}{language=[LaTeX]TeX, classoffset=0, morekeywords={umlbasiccomponent, umlprovidedinterface, umlrequiredinterface, umldelegateconnector, umlassemblyconnector, umlVHVassemblyconnector, umlHVHassemblyconnector, umlnote, umlusecase, umlactor, umlinherit, umlassoc, umlVHextend, umlinclude, umlstateinitial, umlbasicstate, umltrans, umlstatefinal, umlVHtrans, umlHVtrans, umldatabase, umlmulti, umlobject, umlfpart, umlcreatecall, umlclass, umlvirt, umlunicompo, umlimport, umlaggreg}, keywordstyle=\color{DarkBlue}, classoffset=1, morekeywords={umlcomponent, umlsystem, umlstate, umlseqdiag, umlcall, umlcallself, umlfragment, umlpackage}, keywordstyle=\color{DarkRed}, classoffset=0,  sensitive=true, morecomment=[l]{\%}}

\begin{document}
\begin{minipage}[b]{0.55\linewidth}
\Huge \color{NavyBlue} \textbf{OpenST Protocol v0.9.3 } \color{Black}\\ % Title
\huge\textit{[PROPOSED] sequence diagrams for stake and mint - Benjamin Bollen \& Pranay Valson, last edit \today}\\[1cm] % Subtitle
\end{minipage}

\begin{tikzpicture}
  \begin{umlseqdiag}
    \umlactor[class=Address]{Staker}
    \umlactor[class=Worker]{Facilitator}
    \umlactor[class=Worker]{Hunter}
    \umlboundary[class=ERC20]{OST} 
    \umlboundary[class=SK]{Workers} % Worker Contract 
    \umlboundary[class=SK]{Branded Token Gate}
    \umlcontrol[class=SK]{OpenSTValue}
    \umlobject[class=SK]{SimpleStake}
    \umlobject[class=SK]{CoreUC}
    \umlcontrol[class=SK]{RegistrarVC}
    \umlboundary[class=Web3]{Value Chain}
    \umlactor[class=Worker, fill=purple!40]{Foundation}
    \umlboundary[class=Web3, fill=blue!40]{Utility Chain}
    \umlcontrol[class=SK, fill=blue!40]{RegistrarUC}
    \umlobject[class=SK, fill=blue!40]{CoreVC}
    \umlcontrol[class=SK, fill=blue!40]{OpenSTUtility}
    \umlboundary[class=ERC20, fill=blue!40]{Branded Token}
    \umlboundary[class=SK, fill=blue!40]{Token Holder}
  
    %%%
    %%%  Staker initiates stake
    %%% 
    
    % staker approves branded token gate contract on OST for amount
    \begin{umlcall}[op={: approve(gate, amountST)}]{Staker}{OST}  
    \end{umlcall}
          
    % staker requests stake to gate
    \begin{umlcall}[dt=10, op={: requestStake(amountST, gate)}]{Staker}{Branded Token Gate}
      % gate pulls amount from staker on OST
      \begin{umlcall}[fill=green!20, dt=5, op={: transferFrom(staker, gate, amountST)}, return=<<OST(amount)>>]{Branded Token Gate}{OST}
        % pull OST from Facilitator
        \begin{umlcall}[dt=5, fill=green!20, type=return]{Staker}{OST}
        \end{umlcall}
      \end{umlcall}
      
      % emit StakeRequested event which Facilitator listens to
      \begin{umlcall}[dt=5, type=return, op={emit StakeRequested(staker, amountST, gate)}]{Branded Token Gate}{Facilitator}
      \end{umlcall}
    \end{umlcall}
   


    % Facilitator evaluates request against policy (monetary and KYC/AML)
    \begin{umlfragment}[type=alt, label=accept, name=policy, inner xsep=2]
      \begin{umlcall}[dt=10, op={: approve(gate, bounty)}]{Facilitator}{OST}
      \end{umlcall}
      % Facilitator accepts request
      \begin{umlcall}[op={: acceptStakeRequest(staker, hashLock)}]{Facilitator}{Branded Token Gate}  %return={ emit StakingRequestAccpeted (staker, amountST, amountUT, nonce, unlockHeight, stakingIntentHash}
        % Pull Facilitator's bounty into Gate
        \begin{umlcall}[fill=green!20, op={: transferFrom(facilitator, gate, bounty)}, return={<<OST(bounty)>>}]{Branded Token Gate}{OST}
            % pull OST for bounty from Facilitator
            \begin{umlcall}[dt=5, fill=green!20, op={<<OST(bounty)>>},type=return]{Facilitator}{OST}
            \end{umlcall}
        \end{umlcall}
        % Approve OpenSTValue as spender for Gate
        \begin{umlcall}[op={: approve(openSTValue, amountST)}]{Branded Token Gate}{OST}
        \end{umlcall}
        
        % Gate calls on to OpenSTValue to stake (later abstract to library call)
        \begin{umlcall}[op={: stake(uuid, amountST, openSTValue, hashLock, gate)}]{Branded Token Gate}{OpenSTValue} % return={amountUT, nonce, unlockHeight, stakingIntentHash}
          % check stakingAccount 
          \begin{umlcallself}[op={require(msg.sender = gate = stakingAccount)}]{OpenSTValue}
          \end{umlcallself}
          % OpenSTValue pulls amount plus bounty from Facilitator to
          % its OST account balance
          \begin{umlcall}[dt=4, op={: transferFrom(gate, OpenSTValue, amountST)}, fill=green!20, return=<<OST(amount)>>]{OpenSTValue}{OST}
            % pull OST for pre-fund amount from Facilitator
            \begin{umlcall}[dt=20, fill=green!20, type=return]{Branded Token Gate}{OST}
            \end{umlcall}
          \end{umlcall}
          % store StakingIntentHash in contract storage
          \begin{umlcallself}[op={store StakingIntentHash},]{OpenSTValue}
          \end{umlcallself}
          % HTLC(facilitator, amount+bounty)Facilitator
          %\begin{umlcallself}[dt=0, op={HTLC(staker, amount)},]{OpenSTValue}
          %\end{umlcallself}
          % emit StakingIntentDeclared
          \begin{umlcall}[type=return, op={emit StakingIntentDeclared(uuid, gate, nonce, openSTvalue, amountST, amountUT, unlockHeight, StakingIntentHash, ChainIdUtility)}]{OpenSTValue}{Facilitator}
          \end{umlcall}
        \end{umlcall}
        % HTLC(facilitator, bounty)
        %\begin{umlcallself}[op={HTLC(facilitator, bounty)}]{Branded Token Gate}
        %\end{umlcallself}
      \end{umlcall}
      
      % Facilitator rejects request
      \umlfpart[reject]
      \begin{umlcall}[dt=5, op={: rejectRequest(staker, amount)}]{Facilitator}{Branded Token Gate}
        % transfer amount back to staker
        \begin{umlcall}[fill=green!20, op={: transfer(staker, amount)}]{Branded Token Gate}{OST}
          % return OST to staker
          \begin{umlcall}[type=return, fill=green!20, op=<<OST(amount)>>]{OST}{Staker}
          \end{umlcall}
        \end{umlcall}
      \end{umlcall}
    \end{umlfragment}
    \umlnote[x=2, y=-7]{policy}{evaluate request against policy (KYC/AML)}
    \umlnote[x=30,y=-5, width=200]{policy}{to accept the staking request the facilitator generates a secret, random unlockSecret, and publishes hashLock=Hash(unlockSecret)}
    
    % optionally, staker can initiate revert after timeout and no action from Facilitator
    \begin{umlfragment}[type=opt]
      % staker reverts Stake Request after timeout
      \begin{umlcall}[dt=12, op={: revertStakeRequest(staker, amount)}, fill=green!20]{Staker}{Branded Token Gate}
        % gate checks amount is not locked under HTLC or time-lock is not yet expired
        \begin{umlcallself}[op={require(msg.sender=staker and no HTLC)}]{Branded Token Gate}
        \end{umlcallself}
        % transfer amount back to staker
        \begin{umlcall}[fill=green!20, op={: transfer(staker, amount)}]{Branded Token Gate}{OST}
          % return OST to staker
          \begin{umlcall}[type=return, fill=green!20, op=<<OST(amount)>>]{OST}{Staker}
          \end{umlcall}
        \end{umlcall}
      \end{umlcall}
    \end{umlfragment}
    
    %%%
    %%%  OpenST Mosaic (Foundation reports respective state root of )
    %%%
    
    \begin{umlfragment}[type=loop, name=mosaic]
      
      \begin{umlcall}[dt=175, op={new block(blockHeight, stateRoot)}]{Utility Chain}{Foundation}
        % Foundation report state root of Utility chain on CoreUC on value chain
        \begin{umlcall}[op={: report(UC, blockHeight, stateRoot)}]{Foundation}{RegistrarVC}
          \begin{umlcall}[op={: commit(blockHeight, stateRoot)}]{RegistrarVC}{CoreUC}
            % state root of utility chain got reported on value chain
            \begin{umlcall}[dt=10, type=return, op={emit CommittedStateRoot(UC, blockHeight, stateRoot)}]{CoreUC}{Facilitator}
            \end{umlcall}
          \end{umlcall}
        \end{umlcall}
      \end{umlcall}
          
      % Foundation report state root of Value chain on CoreVC on utility chain
      \begin{umlcall}[dt=205, op={new block(blockHeight, stateRoot)}]{Value Chain}{Foundation}
        \begin{umlcall}[op={: report(VC, blockHeight, stateRoot}]{Foundation}{RegistrarUC}
          % reporting through registrar is instant commit
          \begin{umlcall}[op={: commit(blockHeight, stateRoot)}]{RegistrarUC}{CoreVC}
            % state root of value chain got reported on utility chain
            \begin{umlcall}[dt=10, type=return, op={emit CommittedStateRoot(VC, blockHeight, stateRoot)}]{CoreVC}{Facilitator}
            \end{umlcall}
          \end{umlcall}
        \end{umlcall}
      \end{umlcall}
      
    \end{umlfragment}
    \umlnote[x=30, y=-26, width=180]{mosaic}{Placeholder for OpenST Mosaic game}
    
    %%%
    %%% Facilitator has observed a committed state root that includes the StakingIntentHash
    %%%
    
    % Facilitator submits claim for StakingIntentHash by presenting Merkle proof
    \begin{umlcall}[dt=15, op={: claimStakingIntentHash(StakingIntentHash, merkleProof[], committedBlockHeight)}, return={emit ValidatedStakingIntentHash(StakingIntentHash)}]{Facilitator}{OpenSTUtility}
      % OpenSTUtility checks StakingIntentHash against committed state root
      \begin{umlcall}[op={: getStateRoot(blockHeight)}, return={stateRoot @ blockHeight}]{OpenSTUtility}{CoreVC}
      \end{umlcall}
      % OpenSTUtility validate merkle proof
      \begin{umlcallself}[op={validate proof}]{OpenSTUtility}
      \end{umlcallself}
      % OpenSTUtility store valid StakingIntentHash
      \begin{umlcallself}[op={store valid StakingIntentHash}]{OpenSTUtility}
      \end{umlcallself}
    \end{umlcall}
    
    % Facilitator submits pre-image data for StakingIntentHash
    \begin{umlcall}[dt=5, op={: ConfirmStakingIntent(StakingIntentHash, uuid, hashLock, amountST, amountUT, stakingProcessor, stakingProcessorNonce, tokenHolder, unlockHeight)}, return={emit StakingIntentConfirmed(StakingIntentHash, nonce, unlockHeight, expirationHeight)}]{Facilitator}{OpenSTUtility}
      % OpenSTUtility asserts valid pre-image data for StakingIntentHash
      \begin{umlcallself}[op={assert valid pre-image data}]{OpenSTUtility}
      \end{umlcallself}
      % OpenSTUtility checks StakingIntentHash against committed state root
      \begin{umlcall}[op={: getLatestHeight()}, return={latestHeight}]{OpenSTUtility}{CoreVC}
      \end{umlcall}
      % OpenSTUtility asserts grace period before unlockHeight
      \begin{umlcallself}[op={assert grace period before unlockHeight}]{OpenSTUtility}
      \end{umlcallself}
      % OpenSTUtility store mint object with StakingIntentHash and expiration Height
      \begin{umlcallself}[op={store mint with expirationHeight}]{OpenSTUtility}
      \end{umlcallself}
    \end{umlcall}
    
    %%%
    %%% Facilitator has moved both value and utility chain to the first stage
    %%% and can now either proceed or revert by revealing the hash lock secret or await timeout
    %%%
    
    % Correctly initialised staking information on both systems
    \begin{umlfragment}[type=alt, label=proceed, name=phasetwo, inner xsep=6]
      \begin{umlcall}[dt=6, fill=green!20, op={: processStaking(StakingIntentHash, unlockSecret)}]{Facilitator}{Branded Token Gate}
        % check hashlock 
        \begin{umlcallself}[op={require(H(unlockSecret) = hashLock)}]{Branded Token Gate}
        \end{umlcallself}
        % Facilitator first calls processStaking to ensure the bounty is return to him
        \begin{umlcall}[dt=4, op={: processStaking(StakingIntentHash, unlockSecret)}]{Branded Token Gate}{OpenSTValue}
          % check stakingAccount 
          \begin{umlcallself}[op={require(msg.sender = gate = stakingAccount || stakingAccount = 0x)}]{OpenSTValue}
          \end{umlcallself}
          % check hashlock 
          \begin{umlcallself}[op={require(H(unlockSecret) = hashLock)}]{OpenSTValue}
          \end{umlcallself}
          % transfer amount OST to SimpleStake
          \begin{umlcall}[fill=green!20, op={: transfer(simpleStake, amount)}]{OpenSTValue}{OST}
            \begin{umlcall}[dt=5, fill=green!20, type=return, op={<<OST(amount)>>}]{OST}{SimpleStake}
            \end{umlcall}
          \end{umlcall}
          \begin{umlcall}[dt=5, type=return, op={emit ProcessedStake(StakingIntentHash, unlockSecret)}]{OpenSTValue}{Hunter}
          \end{umlcall}
        \end{umlcall}
        % transfer bounty from gate to facilitator
        \begin{umlcall}[dt=5, fill=green!20, op={: transfer(facilitator, bounty)}]{Branded Token Gate}{OST}
          \begin{umlcall}[fill=green!20, type=return, op={<<OST(bounty)>>}]{OST}{Facilitator}
          \end{umlcall}
        \end{umlcall}
      \end{umlcall}
      
      % Facilitator then calls processMinting to mint the utility tokens
      \begin{umlcall}[op={: processMinting(StakingIntentHash, unlockSecret)}]{Facilitator}{OpenSTUtility}
        % check hashlock 
        \begin{umlcallself}[op={require(H(unlockSecret) = hashLock)}]{OpenSTUtility}
        \end{umlcallself}
        % require mint not yet expired
        \begin{umlcallself}[op={require(expirationHeight > block.number)}]{OpenSTUtility}
        \end{umlcallself}
        % mint Branded Tokens
        \begin{umlcall}[fill=green!20, op={: mint(amountUT, tokenHolder)}]{OpenSTUtility}{Branded Token}
          % store claim
          \begin{umlcallself}[op={store claim}]{Branded Token}
          \end{umlcallself}
        \end{umlcall}
        \begin{umlcall}[type=return, op={emit ProcessedMint(StakingIntentHash, unlockSecret)}]{OpenSTUtility}{Hunter}
        \end{umlcall}
      \end{umlcall}
      
      % Facilitator (or anyone) can call claim to transfer UT to token holder
      \begin{umlcall}[dt=5, op={: claim(tokenHolder)}]{Facilitator}{Branded Token}
        \begin{umlcall}[fill=green!20, type=return, op={<<UT(amountUT)>>}]{Branded Token}{Token Holder}
        \end{umlcall}
      \end{umlcall}
      
      %%%
      %%%  Hunter
      %%%
      
      \begin{umlfragment}[type=opt, name=bounty, label={ensure completion}, inner xsep=15]
        % process staking on gate
        \begin{umlcall}[dt=20, fill=green!20, op={: processStaking(StakingIntentHash, unlockSecret)}]{Hunter}{Branded Token Gate}
          % check hashlock 
          \begin{umlcallself}[op={require(H(unlockSecret) = hashLock)}]{Branded Token Gate}
          \end{umlcallself}
          % Facilitator first calls processStaking to ensure the bounty is return to him
          \begin{umlcall}[dt=4, op={: processStaking(StakingIntentHash, unlockSecret)}]{Branded Token Gate}{OpenSTValue}
            % check stakingAccount 
            \begin{umlcallself}[op={require(msg.sender = gate = stakingAccount || stakingAccount = 0x)}]{OpenSTValue}
            \end{umlcallself}
            % check hashlock 
            \begin{umlcallself}[op={require(H(unlockSecret) = hashLock)}]{OpenSTValue}
            \end{umlcallself}
            % transfer amount OST to SimpleStake
            \begin{umlcall}[fill=green!20, op={: transfer(simpleStake, amount)}]{OpenSTValue}{OST}
              \begin{umlcall}[dt=5, fill=green!20, type=return, op={<<OST(amount)>>}]{OST}{SimpleStake}
              \end{umlcall}
            \end{umlcall}
            \begin{umlcall}[dt=5, type=return, op={emit ProcessedStake(StakingIntentHash, unlockSecret)}]{OpenSTValue}{Facilitator}
            \end{umlcall}
          \end{umlcall}
          % transfer bounty from gate to facilitator
          \begin{umlcall}[dt=5, fill=green!20, op={: transfer(hunter, bounty)}]{Branded Token Gate}{OST}
            \begin{umlcall}[fill=green!20, type=return, op={<<OST(bounty)>>}]{OST}{Hunter}
            \end{umlcall}
          \end{umlcall}
        \end{umlcall}
      \end{umlfragment}
      \umlnote[x=12, y=-64, width=150]{bounty}{If stake was left unprocessed, unlock secret is known through mint. Bounty is always transferred to msg.sender of processStaking()}

      \umlfpart[revert]
     
      \begin{umlcall}[dt=20, fill=green!20, op={: revertStaking(StakingIntentHash)}]{Facilitator}{Branded Token Gate}
        % check unlockHeight is in the past
        \begin{umlcallself}[op={require(unlockHeight <= block.number)}]{Branded Token Gate}
        \end{umlcallself}
        % Facilitator reverts stake to get back amount and bounty
        \begin{umlcall}[fill=green!20, op={: revertStaking(StakingIntentHash)}]{Branded Token Gate}{OpenSTValue}
          % check stakingAccount 
          \begin{umlcallself}[op={require(msg.sender = gate = stakingAccount || stakingAccount = 0x)}]{OpenSTValue}
          \end{umlcallself}
          % check unlockHeight
          \begin{umlcallself}[op={require(unlockHeight <= block.number)}]{OpenSTValue}
          \end{umlcallself}
          % transfer amount and bounty OST to facilitator
          \begin{umlcall}[fill=green!20, op={: transfer(staker, amount)}]{OpenSTValue}{OST}
            \begin{umlcall}[fill=green!20, type=return, op={<<OST(amount)>>}]{OST}{Staker}
            \end{umlcall}
          \end{umlcall}
        \end{umlcall}
        % transfer bounty from gate to facilitator
        \begin{umlcall}[dt=5, fill=green!20, op={: transfer(facilitator, bounty)}]{Branded Token Gate}{OST}
          \begin{umlcall}[fill=green!20, type=return, op={<<OST(bounty)>>}]{OST}{Facilitator}
          \end{umlcall}
        \end{umlcall}
      \end{umlcall} 

      \begin{umlcall}[op={: revertMinting(StakingIntentHash)}]{Facilitator}{OpenSTUtility}
        % require mint not yet expired
        \begin{umlcallself}[op={require(expirationHeight <= block.number)}]{OpenSTUtility}
        \end{umlcallself}
      \end{umlcall}
    \end{umlfragment}
    \umlnote[x=1, y=-47, width=100]{phasetwo}{With both value chain and utility chain configured correctly for StakingIntentHash, Facilitator can proceed by revealing the unlock secret, or revert by awaiting the unlock height }
    \umlnote[x=40, y=-100, width=120]{phasetwo}{Any actor can call revertStaking and the bounty is returned to the facilitator}

  \end{umlseqdiag}
\end{tikzpicture}


\end{document}

